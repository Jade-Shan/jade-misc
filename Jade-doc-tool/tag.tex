

\chapter{基础架构}



\section{连接配置模块}

为了建立一个连接需要提供相关的信息,比如:




\begin{figure}[htbp]\centering
					\includegraphics[scale=]{img/des-conn.01.png}\caption{这个是连接的状态}\label{fig:des-conn.01.png}
					\end{figure}



\subsection{连接信息配置:ConnectionConfig}

为了建立一个连接需要提供相关的信息,比如:




\begin{itemize}
	\item   \verb|serviceName| :提供XMPP服务的服务器,可以通过IP地址或是域名指定。
	\item   \verb|port| :服务器运行的服务所监听的端口。
	\item   \verb|username| :登录服务器时需要的用户名。
	\item   \verb|password| :登录服务器时需要的密码。
	\item   \verb|password| :登录服务器时需要的密码。
	\item   \verb|password| :登录服务器时需要的密码。
	\item   \verb|resource| :XMPP协议的Resource,代表了一个连接的客户端。这个连起来

\end{itemize}

这个不应该连起来




\subsubsection{Scala的构造器与成员}

根据上一步  \verb|ConnectionConfig| 类的设计,可以通过Scala语言把它实现出来。在Scala中
要定义一个类非常简单,代码如下:




\begin{lstlisting}[language=Java]
class ConnectionConfiguration
{
	var serviceName: String = null
	var port: Int = null 
	var username: String = null
	var password: String = null 
	var resource: String = null
}

\end{lstlisting}





\section{连接配置模块}



\begin{itemize}
	\item   \verb|serviceName| :提供XMPP服务的服务器,可以通过IP地址或是域名指定。
	\item   \verb|port| :服务器运行的服务所监听的端口。
	\item   \verb|username| :登录服务器时需要的用户名。
	\item   \verb|password| :登录服务器时需要的密码。
	\item   \verb|resource| :XMPP协议的Resource,代表了一个连接的客户端。
\end{itemize}



\subsection{连接信息配置:ConnectionConfig}

为了建立一个连接需要提供相关的信息,比如:






\subsubsection{Scala的构造器与成员}

根据上一步  \verb|ConnectionConfig| 类的设计,可以通过Scala语言把它实现出来。在Scala中
要定义一个类非常简单,代码如下:




\begin{lstlisting}[language=Java]
class ConnectionConfiguration
{
	var serviceName: String = null
	var port: Int = null 
	var username: String = null
	var password: String = null 
	var resource: String = null
}

\end{lstlisting}



